\textbf{property} & \textbf{type} & \textbf{description} & \textbf{default}
\\\hline\hline
\endhead
\texttt{inputField}
& string & The ID of your input field.
& null
\\\hline
\texttt{displayArea}
& string & This is the ID of a $<$span$>$, $<$div$>$, or any other element that you would like to use to display the current date. This is generally useful only if the input field is hidden, as an area to display the date.
& null
\\\hline
\texttt{button}
& string & The ID of the calendar ``trigger''. This is an element (ordinarily a button or an image) that will dispatch a certain event (usually ``click'') to the function that creates and displays the calendar.
& null
\\\hline
\texttt{eventName}
& string & The name of the event that will trigger the calendar. The name should be without the ``on'' prefix, such as ``click'' instead of ``onclick''. Virtually all users will want to let this have the default value (``click''). Anyway, it could be useful if, say, you want the calendar to appear when the input field is focused and have no trigger button (in this case use ``focus'' as the event name).
& ``click''
\\\hline
\texttt{ifFormat}
& string & The format string that will be used to enter the date in the input field. This format will be honored even if the input field is hidden.
& ``\%Y/\%m/\%d''
\\\hline
\texttt{daFormat}
& string & Format of the date displayed in the displayArea (if specified).
& ``\%Y/\%m/\%d''
\\\hline
\texttt{singleClick}
& boolean & Wether the calendar is in ``single-click mode'' or ``double-click mode''. If true (the default) the calendar will be created in single-click mode.
& true
\\\hline
\texttt{disableFunc}
& function & A function that receives a JS Date object.  It should return
\texttt{true} if that date has to be disabled, \texttt{false} otherwise.
{\color{red} DEPRECATED (see below).}
& null
\\\hline
\texttt{dateStatusFunc}
& function & A function that receives a JS Date object and returns a boolean
or a string.  This function allows one to set a certain CSS class to some
date, therefore making it look different.  If it returns \texttt{true} then
the date will be disabled.  If it returns \texttt{false} nothing special
happens with the given date.  If it returns a string then that will be taken
as a CSS class and appended to the date element.  If this string is
``disabled'' then the date is also disabled (therefore is like returning
\texttt{true}).  For more information please also refer to section
\ref{sec:Calendar.setDateStatusHandler}.
& null
\\\hline
\texttt{mondayFirst}
& boolean & If \texttt{true} (default) then the calendar will display with
Monday being the first day of week.  If \texttt{false} then Sunday will be
the first day of week.  This has changed from default \texttt{false} to
default \texttt{true} because the ISO 8601 defines week as starting Monday
and this definition is used for computing the week number.
& true
\\\hline
\texttt{weekNumbers}
& boolean & If ``true'' then the calendar will display week numbers.
& true
\\\hline
\texttt{align}
& string & Alignment of the calendar, relative to the reference element. The
reference element is dynamically chosen like this: if a displayArea is
specified then it will be the reference element. Otherwise, the input field
is the reference element.  For the meaning of the alignment characters
please section \ref{sec:Calendar.showAtElement}.
& ``Bl''
\\\hline
\texttt{range}
& array & An array having exactly 2 elements, integers. (!) The first [0] element is the minimum year that is available, and the second [1] element is the maximum year that the calendar will allow.
& [1900, 2999]
\\\hline
\texttt{flat}
& string & If you want a flat calendar, pass the ID of the parent object in
this property.  If not, pass \texttt{null} here (or nothing at all as
\texttt{null} is the default value).
& null
\\\hline
\texttt{flatCallback}
& function & You should provide this function if the calendar is flat.  It
will be called when the date in the calendar is changed with a reference to
the calendar object.  See section \ref{sec:quick-start-flat} for an example
of how to setup a flat calendar.
& null
\\\hline
\texttt{onSelect}
& function & If you provide a function handler here then you have to manage
the ``click-on-date'' event by yourself.  Look in the calendar-setup.js and
take as an example the onSelect handler that you can see there.
& null
\\\hline
\texttt{onClose}
& function & This handler will be called when the calendar needs to close.
You don't need to provide one, but if you do it's your responsibility to
hide/destroy the calendar.  You're on your own.  Check the calendar-setup.js
file for an example.
& null
\\\hline
\texttt{onUpdate}
& function & If you supply a function handler here, it will be called right
after the target field is updated with a new date.  You can use this to
chain 2 calendars, for instance to setup a default date in the second just
after a date was selected in the first.
& null
\\\hline
\texttt{date}
& date & This allows you to setup an initial date where the calendar will be
positioned to.  If absent then the calendar will open to the today date.
& null
\\\hline
\texttt{showsTime}
& boolean & If this is set to \texttt{true} then the calendar will also
allow time selection.
& false
\\\hline
\texttt{timeFormat}
& string & Set this to ``12'' or ``24'' to configure the way that the
calendar will display time.
& ``24''
\\\hline
\texttt{electric}
& boolean & Set this to ``false'' if you want the calendar to update the
field only when closed (by default it updates the field at each date change,
even if the calendar is not closed) & true
