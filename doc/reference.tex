\documentclass[a4paper,10pt]{article}

% enlarge the page a bit..
\addtolength{\hoffset}{-1cm}
\addtolength{\textwidth}{2cm}
\addtolength{\voffset}{-1.5cm}
\addtolength{\textheight}{3cm}

\usepackage{latexsym}
\usepackage{amssymb}
%\usepackage{tabularx}
\usepackage{ltxtable}
\usepackage{hyperref}

\title{DHTML Calendar Widget}
\author{Mihai Bazon, \texttt{<mishoo@infoiasi.ro>}}
\date{\today}

\begin{document}
\maketitle
\noindent
This document is the API reference to the version 0.9.3 of the DHTML
Calendar widget.  For full source code and latest versions please visit
{\href{http://students.infoiasi.ro/~mishoo/site/calendar.epl}{the calendar
project page}}.

\tableofcontents

% \setlength{\parindent}{0pt}
% \setlength{\parskip}{1.3ex}

\section{Overview}

The DHTML Calendar widget\footnote
        {
        by the term ``widget'' I understand a single element of user interface.
        But that's in Linux world.  For those that did lots of Windows
        programming the term ``control'' might be more familiar
        }
is an (HTML) user interface element that gives end-users a friendly way to
input dates.  It works in a web browser.  The first versions only provided
support for popup calendars, while starting with version 0.9 it also supports
``flat'' display.  A ``flat'' calendar is a calendar that stays visible in the
page all the time.  In this mode it could be very useful for ``blog'' pages and
other pages that require the calendar to be always present.

The calendar is compatible with most popular browsers nowadays.  While it's
created using web standards and it should generally work with any compliant
browser, the following browsers were found to work: Mozilla (the main
development platform), Netscape 6.0 or better, all Gecko-based browsers, Internet Explorer
5.0 or better \emph{for Windows}\footnote{people report that the calendar does
not work with IE5/Mac.  I don't have access to a Macintosh, therefore -- sorry
-- I can't fix it.}, Opera 7\footnote
        { under Opera 7 the calendar still lacks some functionality, such as
        keyboard navigation; also Opera doesn't seem to allow disabling text
        selection when one drags the mouse on the page; despite all that, the
        calendar is still highly functional under Opera 7 and looks as good as
        in other supported browsers. }.

\subsection{How does this thing work?}

DHTML is not ``another kind of HTML''.  It's merely a naming convention.  DHTML
refers to the combination of HTML, CSS, JavaScript and DOM.  DOM (Document
Object Model) is a set of interfaces that glues the other three together.  In
other words, DOM allows dynamic modification of an HTML page through a program.
JavaScript is our programming language, since that's what browsers like.  CSS
is a way to make it look good ;-).

The program dynamically creates a \texttt{<table>} element that contains a
calendar for the given date.  Then it shows this table at a specified position.
Usually the position is related to some element in which the date needs to be
displayed/entered.  By assigning a certain CSS class to the table we can
control the look of the calendar through an external CSS file; therefore, in
order to change the colors, backgrounds, rollover effects and other stuff, you
can only change a CSS file --- modification of the program itself is not
necessary.

\subsection{Project files}

Here's a description of the project files, excluding documentation and example
files.

\begin{itemize}

\item the main program file (\texttt{calendar.js}).  This defines all the logic
behind the calendar widget.

\item the CSS files (\texttt{calendar-*.css}).  Loading one of them is
necessary in order to see the calendar as intended.

\item the language definition files (\texttt{lang/calendar-*.js}).  They are
plain JavaScript files that contain all texts that are displayed by the
calendar.  Loading one of them is necessary.

\item helper functions for quick setup of the calendar
(\texttt{calendar-setup.js}).  You can do fine without it, but starting with
version 0.9.3 this is the recommended way to setup a calendar.

\end{itemize}

\subsection{License}

\begin{center}
\noindent \copyright\ Mihai Bazon, 2002--2003, \texttt{<mishoo@infoiasi.ro>}\\
\href{http://students.infoiasi.ro/~mishoo/site/calendar.epl}{\texttt{http://students.infoiasi.ro/\~{}mishoo/site/calendar.epl}}\\
\end{center}

The calendar is released under the
{\href{http://www.gnu.org/licenses/lgpl.html}{GNU Lesser General Public
License}}.  This basically means that you are allowed to use it for anything you
like, except selling it for profit or claiming it's authorship.  You can read
the entire license text {\href{http://www.gnu.org/licenses/lgpl.html}{here}}.




\section{Quick startup}

Installing the calendar used to be quite a task until version 0.9.3.  Starting
with 0.9.3 I have included the file \texttt{calendar-setup.js} whose goal is to
assist you to setup a popup or flat calendar within minutes.

First you have to include the needed scripts and stylesheet.  Make sure you do
this in your document's \texttt{<head>} section, also make sure you put the
correct paths to the scripts.

\begin{verbatim}
<style type="text/css">@import url(calendar-win2k-1.css)</style>
<script type="text/javascript" src="calendar.js"></script>
<script type="text/javascript" src="lang/calendar-en.js"></script>
<script type="text/javascript" src="calendar-setup.js"></script>
\end{verbatim}

\subsection{Installing a popup calendar}\label{sec:quick-start-popup}

\noindent Now suppose you have the following HTML:

\begin{verbatim}
<form ...>
  <input type="text" id="data" name="data" />
  <button id="trigger">...</button>
</form>
\end{verbatim}

\noindent You want the button to popup a calendar widget when clicked?  Just
insert the following code immediately \emph{after} the HTML form:

\begin{verbatim}
<script type="text/javascript">
  Calendar.setup(
    {
      inputField  : "data",      // ID of the input field
      ifFormat    : "M d, y",    // the date format
      button      : "trigger"    // ID of the button
    }
  );
</script>
\end{verbatim}

The \texttt{Calendar.setup} function, defined in \texttt{calendar-setup.js}
takes care of ``patching'' the button to display a calendar when clicked.  The
calendar is by default in single-click mode and linked with the given input
field, so that when the end-user selects a date it will update the input field
with the date in the given format and close the calendar.  If you are a
long-term user of the calendar you probably remember that for doing this you
needed to write a couple functions and add an ``onclick'' handler for the
button by hand.

By looking at the example above we can see that the function
\texttt{Calendar.setup} receives only one parameter: a JavaScript object.
Further, that object can have lots of properties that tell to the setup
function how would we like to have the calendar.  For instance, if we would
like a calendar that closes at double-click instead of single-click we would
also include the following: \texttt{singleClick:false}.

For a list of all supported parameters please see the section
\ref{sec:Calendar.setup}.

\subsection{Installing a flat calendar}\label{sec:quick-start-flat}

Here's how to configure a flat calendar, using the same \texttt{Calendar.setup}
function.  First, you should have an empty element with an ID.  This element
will act as a container for the calendar.  It can be any block-level element,
such as DIV, TABLE, etc.  We will use a DIV in this example.

\begin{verbatim}
<div id="calendar-container"></div>
\end{verbatim}

Then there is the JavaScript code that sets up the calendar into the
``calendar-container'' DIV.  The code can occur anywhere in HTML
\emph{after} the DIV element.

\begin{verbatim}
<script type="text/javascript">
  function dateChanged(calendar) {
    // Beware that this function is called even if the end-user only
    // changed the month/year.  In order to determine if a date was
    // clicked you can use the dateClicked property of the calendar:
    if (calendar.dateClicked) {
      // OK, a date was clicked, redirect to /yyyy/mm/dd/index.php
      var y = calendar.date.getFullYear();
      var m = calendar.date.getMonth();     // integer, 0..11
      var d = calendar.date.getDate();      // integer, 1..31
      // redirect...
      window.location = "/" + y + "/" + m + "/" + d + "/index.php";
    }
  };

  Calendar.setup(
    {
      flat         : "calendar-container", // ID of the parent element
      flatCallback : dateChanged           // our callback function
    }
  );
</script>
\end{verbatim}

\subsection{\texttt{Calendar.setup} in detail}\label{sec:Calendar.setup}

Folloing there is the complete list of properties interpreted by
Calendar.setup.  All of them have default values, so you can pass only those
which you would like to customize.  Anyway, you \emph{must} pass at least one
of \texttt{inputField}, \texttt{displayArea} or \texttt{button}.  Otherwise you
will get a warning message saying that there's nothing to setup.

\begin{small}
\ifx\shipout\undefined
\input{Calendar.setup.html.tex}
\else
\LTXtable{\textwidth}{Calendar.setup.pdf.tex}
\fi
\end{small}




\section{The Calendar object}

The file \texttt{calendar.js} implements the functionality of the calendar.
All (well, almost all) functions and variables are embedded in the JavaScript
object ``Calendar''.

``TO BE WRITTEN'' [FIXME].

\subsection{\texttt{Calendar.showAtElement}}\label{sec:Calendar.showAtElement}

This function is useful if you want to display the calendar near some element.
You call it like this:

\begin{verbatim}
calendar.showAtElement(element, align);
\end{verbatim}

\noindent where element is a reference to your element (for instance it can be the input
field that displays the date) and align is an optional parameter, of type string,
containing one or two characters.  For instance, if you pass \texttt{"Br"} as
align, the calendar will appear \emph{below} the element and with its right
margin continuing the element's right margin.

As stated above, align may contain one or two characters.  The first character
dictates the vertical alignment, relative to the element, and the second
character dictates the horizontal alignment.  If the second character is
missing it will be assumed \texttt{"l"} (the left margin of the calendar will
be at the same horizontal position as the left margin of the element).

The characters given for the align parameters are case sensitive.

\subsubsection{Vertical alignment}

The first character in ``\texttt{align}'' can take one of the following values:

\begin{itemize}

\item \texttt{T} -- completely above the reference element (bottom margin of
the calendar aligned to the top margin of the element).

\item \texttt{t} -- above the element but may overlap it (bottom margin of the calendar aligned to
the bottom margin of the element).

\item \texttt{c} -- the calendar displays vertically centered to the reference
element.  It might overlap it (that depends on the horizontal alignment).

\item \texttt{b} -- below the element but may overlap it (top margin of the calendar aligned to
the top margin of the element).

\item \texttt{B} -- completely below the element (top margin of the calendar
aligned to the bottom margin of the element).

\end{itemize}

\subsubsection{Horizontal alignment}

The second character in ``\texttt{align}'' can take one of the following values:

\begin{itemize}

\item \texttt{L} -- completely to the left of the reference element (right
margin of the calendar aligned to the left margin of the element).

\item \texttt{l} -- to the left of the element but may overlap it (left margin
of the calendar aligned to the left margin of the element).

\item \texttt{c} -- horizontally centered to the element.  Might overlap it,
depending on the vertical alignment.

\item \texttt{r} -- to the right of the element but may overlap it (right
margin of the calendar aligned to the right margin of the element).

\item \texttt{R} -- completely to the right of the element (left margin of the
calendar aligned to the right margin of the element).

\end{itemize}

\subsubsection{Default values}

If the ``\texttt{align}'' parameter is missing the calendar will choose
``\texttt{Bl}''.  This resembles the behavior of older versions (prior to
0.9.3) which did not support custom alignment.




\section{Thank you}

A final word, I wish to thank to people that submitted bug reports, feature
requests, suggestions, language definition files, etc.

\begin{quote}
Special thanks to Jon Stokkeland ({\href{http://sauen.com}{Sauen.com}}) for
buying a commercial development license.  I actually see this as a donation ;-)
\end{quote}

Donations can greatly increase my interest in continuing the development.  I
like doing free programs, but as I came to notice, they bring no money but more
work ;-).  If you wish to donate some money, regardless the amount, please
contact me at \texttt{<mishoo@infoiasi.ro>}.

\end{document}
